% !TeX root = ../main.tex

\section{Matlab}

\begin{lstlisting}[language=Matlab, escapeinside=`']
    close all
    clearvars
    clc
\end{lstlisting}

\subsection{Cosine signals}

\subsubsection{Sum of multiple signals, lcm}
The signal $w(n)$ is defiend as the sum of 3 cosinusoidal signals, where the first signal has frequency $f_0$, the second $f_0/2$, the third $f_0/4$. Define $w(n)$ such that it repeats periodically every 10ms, knowing that it is sampled with $F_s=1kHz$
\begin{lstlisting}[language=Matlab, escapeinside=`']
    Fs = 1e3;
    P = 0.01;
    P_samples = P*Fs;

    % find the least common multiple among 1/f0, 2/f0, 4/f0
    % --> 4/f0 = P_samples --> f0 = 4/P_samples
    f0 = 4/P_samples;
    f1 = f0/2;
    f2 = f0/4;
    N = 10*P_samples;
    n = 0:N-1;
    w = cos(2*pi*f0*n) + cos(2*pi*f1*n) + cos(2*pi*f2*n);
\end{lstlisting}

\subsection{Filters}

\subsubsection{FIR filter to attenuate signal and conjugate zero/pole}
You are given two zeros with absolute value equal to 2 in complex conjugate position, build a FIR filter in order to attenuate a signal with frequency $f_1=\cdots$
\begin{lstlisting}[language=Matlab, escapeinside=`']
    z1 = 2*exp(1i*2*pi*f1);
    z2 = conj(z1);

    A = 1; % it is a FIR
    % If we have two zeroes which are complex conjugate, we can
    % always define the polynomial related to the zeros as
    % 1 - 2*rho*cos(theta)z^{-1} + rho^2 z^{-2}
    B = [1, -2*cos(2*pi*f1), 4];
    y = filter(B,A,x);
\end{lstlisting}
