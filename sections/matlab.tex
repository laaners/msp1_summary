% !TeX root = ../main.tex

\section{Matlab}

\begin{lstlisting}[language=Matlab, escapeinside=`']
    close all
    clearvars
    clc
\end{lstlisting}

\subsection{Cosine signals}

    \subsubsection{Sinusoid signal continuous, in time and in samples plot}
    A sinusoid in continuous domain is
    $$
    x(t)=A\cdot\cos(2\pi ft+\phi)
    $$
    Its sampled version is:
    $$
    \begin{cases}
        x(n)=A\cdot\cos(2\pi \tilde{f}n+\phi)\\
        \tilde{f}=\frac{f}{F_s}
    \end{cases}
    $$
    Given:
    $$
    \begin{cases}
        t=[0,0.5]\\
        F_s=1000\,\,Hz=\frac{1}{T_s}\qquad\text{sampling rate}\\
        A=0.8\qquad\text{amplitude}\\
        f=50\,\,Hz\qquad\text{frequency, not normalized}\\
        \phi=30^\circ=\frac{\pi}{6}\qquad\text{phase}
    \end{cases}
    $$
    In matlab:
    \begin{lstlisting}[language=Matlab, escapeinside=`']
        A = 0.8;
        f = 50;
        Fs = 1e3; %1000;
        Ts = 1/Fs;
        t = 0:Ts:0.5;
        phi = 30; % deg
        
        % conversion
        % phi_deg:phi_rad = 180: pi
        phi_rad = phi*pi/180;
        
        x = A*cos(2*pi*f*t + phi_rad);

        %% Plot the signals as a function of time and as a function of samples

        figure(1);
        plot(t, x);
        hold on;
        plot(t, x1, '--');
        title('Signals as a function of time');

        figure(2);
        plot(x); %plot(1:length(x), x);
        hold on;
        plot(1:length(x1), x1, '--');
        title('Signals as a function of samples');

    \end{lstlisting}

    \subsubsection{Sum of multiple signals continuous, lcm}
    Build a signal $x(n)$ as the sum of three different sinusoids  at the normalized angular frequencies $\tilde{\omega}_1=\pi/5$, $\tilde{\omega}_2=\pi/8$, $\tilde{\omega}_3=\pi/4$.\\
    The sampling period is $T_s = 0.3$ seconds, and  the signal is defined for $t$ in $[0, 100]$ seconds.

    We know that
    $$
    \tilde{\omega}=2\pi\tilde{f}=\frac{\omega}{F_s}=\omega T_s
    $$
    \begin{lstlisting}[language=Matlab, escapeinside=`']
        T = .3;
        end_duration = 100;
        time = 0:T:end_duration;
        
        % normalized omegas
        omega_1_n = pi/5; 
        omega_2_n = pi/8;
        omega_3_n = pi/4;
        
        % omegas 
        omega_1 = omega_1_n / T;
        omega_2 = omega_2_n / T;
        omega_3 = omega_3_n / T;
        
        % principle of superposition
        A_1 = 1; % not specified so 1
        A_2 = 1; % not specified so 1
        A_3 = 1; % not specified so 1
        x = A_1*cos(omega_1*time) + A_2*cos(omega_2*time) + A_3*cos(omega_3*time);
    \end{lstlisting}
    For the period of sinusoid in time and samples we have:
    $$
    \frac{1}{f}=\frac{2\pi T_s}{\tilde{\omega}}\qquad\text{Period in seconds}
    $$
    $$
    \frac{1}{f\cdot T_s}=\frac{1}{\tilde{f}}=\frac{2\pi}{\tilde{\omega}}=\frac{F_s}{f}\qquad\text{Period in samples}
    $$
    \begin{lstlisting}[language=Matlab, escapeinside=`']
        %% compute the period of the sinusoids (seconds)

        P_1 = 2*pi*T/omega_1_n;
        P_2 = 2*pi*T/omega_2_n;
        P_3 = 2*pi*T/omega_3_n;
        
        % If you work with matrices: NB: here you have to put ./ otherwise MATLAB reports an error.
        % P = 2*pi*T ./ Omega_n;s
        
        %% compute the period of the sinusoids (samples)
        
        P_1_samples = P_1 /T; % 10
        P_2_samples = P_2 /T; % 16
        P_3_samples = P_3 /T; % 8
        
        %% period of x = lcm among (10, 16, 8) = 2^4 * 5
        
        P_x_samples = lcm(lcm(P_1_samples, P_2_samples), P_3_samples);
        P_x = P_x_samples * T;
    \end{lstlisting}

    \subsubsection{Sum of multiple signals discrete, lcm}
    The signal $w(n)$ is defiend as the sum of 3 cosinusoidal signals, where the first signal has frequency $f_0$, the second $f_0/2$, the third $f_0/4$. Define $w(n)$ such that it repeats periodically every 10ms, knowing that it is sampled with $F_s=1kHz$
    \begin{lstlisting}[language=Matlab, escapeinside=`']
        Fs = 1e3;
        P = 0.01;
        P_samples = P*Fs;

        % find the least common multiple among 1/f0, 2/f0, 4/f0
        % --> 4/f0 = P_samples --> f0 = 4/P_samples
        f0 = 4/P_samples;
        f1 = f0/2;
        f2 = f0/4;
        N = 10*P_samples;
        n = 0:N-1;
        w = cos(2*pi*f0*n) + cos(2*pi*f1*n) + cos(2*pi*f2*n);
    \end{lstlisting}

\subsection{Discrete Signals (non-sinusoidal)}

    \subsubsection{Shifting}
    Signal $x(n)=0.8^nu(n)$ in $n=1:20$. Generate the signals $y_1(n)=x(n-5)$ and $y_2(n)=x(n+5)$ always in $n=1:20$
    \begin{lstlisting}[language=Matlab, escapeinside=`']
        % Generate the signal x(n) = (0.8)^n u(n), n = 1:20
        % Generate the signal y1(n) = x(n-5),  n = 1:20
        % Generate the signal y2(n) = x(n+5), n = 1:20
        n = 1:20;
        x = (0.8).^n;
        
        % initialize the two signals
        y1 = zeros(size(n));
        y2 = zeros(size(x));

        y1 = circshift(x, 5); % x(n-5)
        y1(1:5) = 0;

        y2 = circshift(x, -5); % x(n+5)
        y2(end-5:end) = 0;
    \end{lstlisting}

    \subsubsection{Periodicity}
    Generate the signal $x(n)=u(n-5)-u(n-10)$ considering $n=1:15$, then generate the periodic version $x_p(n)$ wih period $N=15$ considering $n=1:200$. Then plot the periodic signal considering only 8 periods.
    \begin{lstlisting}[language=Matlab, escapeinside=`']
        % generate signal x(n)
        N = 15; % period
        n = 1:N;
        x = zeros(1,N);
        x(n >= 5 & n < 10) = 1;
        stem(n, x);
        hold on;

        % Generate the periodic signal xp(n) with period N = 15,
        % considering n = 1:200

        n_max = 200;
        N_p = ceil(n_max/N);
        x_p = repmat(x,1,N_p);
        x_p = x_p(1:n_max);

        stem(1:15*8, x_p(1:15*8));
    \end{lstlisting}

    \subsubsection{Convolution}
    Given:

    $
    x(n)=[3,11,7,0,-1,4,2]\qquad n\in[-3,3]\\
    h(n)=[2,3,0,-5,2,1]\qquad n\in[-1,4]
    $

    Compute the convolution
    \begin{lstlisting}[language=Matlab, escapeinside=`']
        n_x = -3:3;
        n_h = -1:4;
        
        x = [3, 11, 7, 0, -1, 4, 2];
        h = [2, 3, 0, -5, 2, 1];
        y = conv(x, h);
        supp = n_x(1) + n_h(1):n_x(end) + n_h(end); % always like this

        figure;
        stem(supp, y);
    \end{lstlisting}

    \subsubsection{Shifting through convolution}
    Given $x(n)=[3,11,7,0,-1,4,2]$, $n\in[-3,3]$, create $y(n)=x(n-5)$ $n\in[0,10]$ without using circshift or for loops
    \begin{lstlisting}[language=Matlab, escapeinside=`']
        % y(n)=x(n-5)=x(n)*\delta(n-5)
        n_x = -3:3;
        x = [3, 11, 7, 0, -1, 4, 2];
        
        n_h = 0:10;
        delta_5 = zeros(size(n_h));
        delta_5(n_h == 5) = 1;
        
        % support of the convolution
        n_conv = n_x(1) + n_h(1):n_x(end) + n_h(end);
        
        y = conv(delta_5, x); % as commutative
        
        % but we want only from 0:10
        y = y(n_conv >= 0 & n_conv <= 10);
        stem(0:10, y);
    \end{lstlisting}
    Create $y(n)=\frac{1}{3}\sum_{m=0}^2x(n-m)$ $n\in[0,10]$ without using circshift or for loops
    \begin{lstlisting}[language=Matlab, escapeinside=`']
        % define the filter, which is 1/3 (delta(n) + delta(n-1) + delta(n-2))
        n_h = 0:10;
        h = zeros(size(n_h));
        h(1:3) = 1/3;
        
        % support of the convolution:
        n_conv = n_x(1) + n_h(1): n_x(end) + n_h(end);
        
        y = conv(x, h);
        
        % we wanted y defined only for n = 0:10 
        y = y(n_conv>=0 & n_conv <= 10);
    \end{lstlisting}

\subsection{Z transform}

\subsection{Filters}

    \subsubsection{FIR filter to attenuate signal and conjugate zero/pole}
    You are given two zeros with absolute value equal to 2 in complex conjugate position, build a FIR filter in order to attenuate a signal with frequency $f_1=\cdots$
    \begin{lstlisting}[language=Matlab, escapeinside=`']
        z1 = 2*exp(1i*2*pi*f1);
        z2 = conj(z1);

        A = 1; % it is a FIR
        % If we have two zeroes which are complex conjugate, we can
        % always define the polynomial related to the zeros as
        % 1 - 2*rho*cos(theta)z^{-1} + rho^2 z^{-2}
        B = [1, -2*cos(2*pi*f1), 4];
        y = filter(B,A,x);
    \end{lstlisting}

\subsection{Functions Recap}
\begin{lstlisting}[language=Matlab, escapeinside=`']
    %==========================================================================
    %% Shifting discrete signals, a positive value will shift to the right. Circular, if shifting right by n, first n values will become the last n values
    circshift([1 2 3 4 5], 2);
    % [4 5 3 2 1]
    %==========================================================================
    %% Periodic sequence generation. The first paramater is the matrix, the second one is the rows repetition, the third is the cols repetition
    repmat([1 2 3],1,3);
    % [1 2 3 1 2 3 1 2 3]
    repmat([1 2 3],2,3);
    % [1 2 3 1 2 3 1 2 3;
    %  1 2 3 1 2 3 1 2 3] 
    %==========================================================================
    %% Returns the convolution of vectors u and v. If u and v are vectors of polynomial coefficients, convolving them is equivalent to multiplying the two polynomials.
    conv([1 1],[1 1]) % (1+x)(1+x)
    % [1 2 1], which is (1+x)(1+x)=1+2x+x^2
    %==========================================================================
    %==========================================================================
    %==========================================================================
    %==========================================================================
    %==========================================================================
    %==========================================================================
    %==========================================================================
\end{lstlisting}
