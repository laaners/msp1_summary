% !TeX root = ../main.tex

\section{Matlab}

\begin{lstlisting}[language=Matlab, escapeinside=`']
    close all
    clearvars
    clc
\end{lstlisting}

\subsection{Sinusoid Signals}

    \subsubsection{Sinusoid signal continuous, in time and in samples plot}
    A sinusoid in continuous domain is
    $$
    x(t)=A\cdot\cos(2\pi ft+\phi)
    $$
    Its sampled version is:
    $$
    \begin{cases}
        x(n)=A\cdot\cos(2\pi \tilde{f}n+\phi)\\
        \tilde{f}=\frac{f}{F_s}
    \end{cases}
    $$
    Given:
    $$
    \begin{cases}
        t=[0,0.5]\\
        F_s=1000\,\,Hz=\frac{1}{T_s}\qquad\text{sampling rate}\\
        A=0.8\qquad\text{amplitude}\\
        f=50\,\,Hz\qquad\text{frequency, not normalized}\\
        \phi=30^\circ=\frac{\pi}{6}\qquad\text{phase}
    \end{cases}
    $$
    In matlab:
    \begin{lstlisting}[language=Matlab, escapeinside=`']
        A = 0.8;
        f = 50;
        Fs = 1e3; %1000;
        Ts = 1/Fs;
        duration = 0.5:
        t = 0:Ts:duration;
        t = 0:1/Fs:duration;
        phi = 30; % deg
        
        % conversion
        % phi_deg:phi_rad = 180: pi
        phi_rad = phi*pi/180;
        
        x = A*cos(2*pi*f*t + phi_rad);

        %% Plot the signals as a function of time and as a function of samples

        figure(1);
        plot(t, x);
        hold on;
        plot(t, x1, '--');
        title('Signals as a function of time');

        figure(2);
        plot(x); %plot(1:length(x), x);
        hold on;
        plot(1:length(x1), x1, '--');
        title('Signals as a function of samples');

    \end{lstlisting}

    \subsubsection{Sum of multiple signals continuous, lcm}
    Build a signal $x(n)$ as the sum of three different sinusoids  at the normalized angular frequencies $\tilde{\omega}_1=\pi/5$, $\tilde{\omega}_2=\pi/8$, $\tilde{\omega}_3=\pi/4$.\\
    The sampling period is $T_s = 0.3$ seconds, and  the signal is defined for $t$ in $[0, 100]$ seconds.

    We know that
    $$
    \tilde{\omega}=2\pi\tilde{f}=\frac{\omega}{F_s}=\omega T_s
    $$
    \begin{lstlisting}[language=Matlab, escapeinside=`']
        T = .3;
        end_duration = 100;
        time = 0:T:end_duration;
        
        % normalized omegas
        omega_1_n = pi/5; 
        omega_2_n = pi/8;
        omega_3_n = pi/4;
        
        % omegas 
        omega_1 = omega_1_n / T;
        omega_2 = omega_2_n / T;
        omega_3 = omega_3_n / T;
        
        % principle of superposition
        A_1 = 1; % not specified so 1
        A_2 = 1; % not specified so 1
        A_3 = 1; % not specified so 1
        x = A_1*cos(omega_1*time) + A_2*cos(omega_2*time) + A_3*cos(omega_3*time);
    \end{lstlisting}
    For the period of sinusoid in time and samples we have:
    $$
    \frac{1}{f}=\frac{2\pi T_s}{\tilde{\omega}}\qquad\text{Period in seconds}
    $$
    $$
    \frac{1}{f\cdot T_s}=\frac{1}{\tilde{f}}=\frac{2\pi}{\tilde{\omega}}=\frac{F_s}{f}\qquad\text{Period in samples}
    $$
    \begin{lstlisting}[language=Matlab, escapeinside=`']
        %% compute the period of the sinusoids (seconds)

        P_1 = 2*pi*T/omega_1_n;
        P_2 = 2*pi*T/omega_2_n;
        P_3 = 2*pi*T/omega_3_n;
        
        % If you work with matrices: NB: here you have to put ./ otherwise MATLAB reports an error.
        % P = 2*pi*T ./ Omega_n;s
        
        %% compute the period of the sinusoids (samples)
        
        P_1_samples = P_1 /T; % 10
        P_2_samples = P_2 /T; % 16
        P_3_samples = P_3 /T; % 8
        
        %% period of x = lcm among (10, 16, 8) = 2^4 * 5
        
        P_x_samples = lcm(lcm(P_1_samples, P_2_samples), P_3_samples);
        P_x = P_x_samples * T;
    \end{lstlisting}

    \subsubsection{Sum of multiple signals discrete, lcm}
    The signal $w(n)$ is defiend as the sum of 3 cosinusoidal signals, where the first signal has frequency $f_0$, the second $f_0/2$, the third $f_0/4$. Define $w(n)$ such that it repeats periodically every 10ms, knowing that it is sampled with $F_s=1kHz$
    \begin{lstlisting}[language=Matlab, escapeinside=`']
        Fs = 1e3;
        P = 0.01;
        P_samples = P*Fs;

        % find the least common multiple among 1/f0, 2/f0, 4/f0
        % --> 4/f0 = P_samples --> f0 = 4/P_samples
        f0 = 4/P_samples;
        f1 = f0/2;
        f2 = f0/4;
        N = 10*P_samples;
        n = 0:N-1;
        w = cos(2*pi*f0*n) + cos(2*pi*f1*n) + cos(2*pi*f2*n);
    \end{lstlisting}

\pagebreak\subsection{Discrete Signals (non-sinusoidal)}

    \subsubsection{Shifting}
    Signal $x(n)=0.8^nu(n)$ in $n=1:20$. Generate the signals $y_1(n)=x(n-5)$ and $y_2(n)=x(n+5)$ always in $n=1:20$
    \begin{lstlisting}[language=Matlab, escapeinside=`']
        % Generate the signal x(n) = (0.8)^n u(n), n = 1:20
        % Generate the signal y1(n) = x(n-5),  n = 1:20
        % Generate the signal y2(n) = x(n+5), n = 1:20
        n = 1:20;
        x = (0.8).^n;
        
        % initialize the two signals
        y1 = zeros(size(n));
        y2 = zeros(size(x));

        y1 = circshift(x, 5); % x(n-5)
        y1(1:5) = 0;

        y2 = circshift(x, -5); % x(n+5)
        y2(end-5:end) = 0;
    \end{lstlisting}

    \subsubsection{Periodicity}
    Generate the signal $x(n)=u(n-5)-u(n-10)$ considering $n=1:15$, then generate the periodic version $x_p(n)$ wih period $N=15$ considering $n=1:200$. Then plot the periodic signal considering only 8 periods.
    \begin{lstlisting}[language=Matlab, escapeinside=`']
        % generate signal x(n)
        N = 15; % period
        n = 1:N;
        x = zeros(1,N);
        x(n >= 5 & n < 10) = 1;
        stem(n, x);
        hold on;

        % Generate the periodic signal xp(n) with period N = 15,
        % considering n = 1:200

        n_max = 200;
        N_p = ceil(n_max/N);
        x_p = repmat(x,1,N_p);
        x_p = x_p(1:n_max);

        stem(1:15*8, x_p(1:15*8));
    \end{lstlisting}

    \subsubsection{Convolution}
    Given:

    $
    x(n)=[3,11,7,0,-1,4,2]\qquad n\in[-3,3]\\
    h(n)=[2,3,0,-5,2,1]\qquad n\in[-1,4]
    $

    Compute the convolution
    \begin{lstlisting}[language=Matlab, escapeinside=`']
        n_x = -3:3;
        n_h = -1:4;
        
        x = [3, 11, 7, 0, -1, 4, 2];
        h = [2, 3, 0, -5, 2, 1];
        y = conv(x, h);
        supp = n_x(1) + n_h(1):n_x(end) + n_h(end); % always like this

        figure;
        stem(supp, y);
    \end{lstlisting}

    \subsubsection{Shifting through convolution}
    Given $x(n)=[3,11,7,0,-1,4,2]$, $n\in[-3,3]$, create $y(n)=x(n-5)$ $n\in[0,10]$ without using circshift or for loops
    \begin{lstlisting}[language=Matlab, escapeinside=`']
        % y(n)=x(n-5)=x(n)*\delta(n-5)
        n_x = -3:3;
        x = [3, 11, 7, 0, -1, 4, 2];
        
        n_h = 0:10;
        delta_5 = zeros(size(n_h));
        delta_5(n_h == 5) = 1;
        
        % support of the convolution
        n_conv = n_x(1) + n_h(1):n_x(end) + n_h(end);
        
        y = conv(delta_5, x); % as commutative
        
        % but we want only from 0:10
        y = y(n_conv >= 0 & n_conv <= 10);
        stem(0:10, y);
    \end{lstlisting}
    Create $y(n)=\frac{1}{3}\sum_{m=0}^2x(n-m)$ $n\in[0,10]$ without using circshift or for loops
    \begin{lstlisting}[language=Matlab, escapeinside=`']
        % define the filter, which is 1/3 (delta(n) + delta(n-1) + delta(n-2))
        n_h = 0:10;
        h = zeros(size(n_h));
        h(1:3) = 1/3;
        
        % support of the convolution:
        n_conv = n_x(1) + n_h(1): n_x(end) + n_h(end);
        
        y = conv(x, h);
        
        % we wanted y defined only for n = 0:10 
        y = y(n_conv>=0 & n_conv <= 10);
    \end{lstlisting}

\pagebreak\subsection{Z-transform}

    \subsubsection{Convolution as product in frequency domain}
    We can use conv here as well, even though the z-transform of a convolution is the product, the coefficients of the resulting polynomial are the same of the coefficients of the time domain convolution
    \begin{lstlisting}[language=Matlab, escapeinside=`']
        n_x = -2:2;
        x = [3, 2, 1, 0, 1];
        
        n_h = 0:4;
        h = [1, 3, 2.5, 4, 2];
        
        % support of the convolution -2:6
        n_y = n_x(1) + n_h(1):n_x(end) + n_h(end);
        
        y = conv(h, x); % as commutative
        
        % Write the expression of H(z)
        % 1\delta(n)+3\delta(n-1)+2.5\delta(n-2)+4\delta(n-3)+2\delta(n-4)
        % H(z) = 1+3z^-1+2.5z^-2+4z^-3+2z^-4
        H_z = h; % we just insert the coefficients
        
        % analog way for X
        % X(z)=3z^2+2z+1+z^-2
        X_z = x;
        
        % convolution in z-domain is just product, how do we do it?
        % Just ignore polynomial product
        % Due to convolution theorem and the fact that final support
        % Y(z)=y0z^2+...+yNz^-6
        % starts from 2 as support in -2:6.*-1 = 2:-6
        % Where those coefficients are the same found from conv, so:
        Y_z = y;
    \end{lstlisting}

    \subsubsection{Filter cascade}
    $$
    H(z)=h_0\cdot H_1(z)\cdot H_2(z)\cdots H_k(z)
    $$
    $$
    \Downarrow
    $$
    $$
    h(n)=h_0\cdot h_1(n)*h_2(n)*h_3(n)*\cdots*h_k(n)
    $$
    We express the filter cascade in the frequency domain
    \begin{lstlisting}[language=Matlab, escapeinside=`']
        n_h = 0:4;
        h = [1, 3, 2.5, 4, 2];
        h_roots = roots(H_z);

        % For h0 just put when support == 0
        h_0 = h(n_h == 0); % [1 0 0 0 0]
        
        % Now for every root 1-z_iz^{-1}
        for r = 1:length(h_roots)
            h_r = [1, -h_roots(r)]; % coefficient of this sub H_i
            % The support will change, as we go on convolution contribtion
            if r == 1
                h_cascade = h_r;
                sup_cascade = [0, 1];
            else
                % convolve by the cascade
                h_cascade = conv(h_r, h_cascade);
                % new support of the cascade
                sup_cascade = sup_cascade(1) + 0:sup_cascade(end) + 1;
            end
        end
        
        % final cascade
        h_cascade = h_0 * h_cascade;
        y = conv(x, h_cascade);
    \end{lstlisting}

    \subsubsection{Partial fract expansion of z-transform}
    Givan a LTI system:
    $$
    H(z)=\frac{
        z^{-5}+z^{-4}-3z^{-3}-8z^{-2}+7z^{-1}+9
    }{
        z^{-3}-2z^{-2}-z^{-1}+2
    }
    $$
    Find its partial fract expansion and then find $h(n)$ as the sum of the elementary filters found with the partial fract expansion, $n=0:100$.
    
    A reminder:
    $$
    H(z)=\frac{Y(z)}{X(z)}=\frac{\sum_{k=0}^Nb_kz^{-k}}{\sum_{k=0}^Da_kz^{-k}}
    =
    \sum_{k=1}^D\sum_{m=1}^M\frac{
        r_{k_m}
    }{(1-p_kz^{-1})^m}+
    \underlabel{
        \sum_{k=0}^{N-D}c_kz^{-k}
    }{$N\geq D$}
    $$
    \begin{itemize}
        \item $Z^{-1}\begin{Bmatrix}
            \frac{r_{k_1}}{(1-p_kz^{-1})}
        \end{Bmatrix}=r_{k_1}\cdot (p_k)^nu(n)$
        \item $Z^{-1}\begin{Bmatrix}
            \frac{r_{k_2}}{(1-p_kz^{-1})^2}
        \end{Bmatrix}=r_{k_2}\cdot (n+1)(p_k)^nu(n)$
        \item $Z^{-1}\Brackets{c_kz^{-k}}=c_k\cdot \delta(n-k)$
    \end{itemize}
    \begin{lstlisting}[language=Matlab, escapeinside=`']
        A_z = [2 -1 -2 1];
        B_z = [9 7 -8 -3 1 1];
        
        [r, p, c] = residuez(B_z, A_z); % residuez!!!
        
        n = 0:100;
        h_partial = zeros(size(n));
        
        h_partial(1:length(c))= c;
        
        for r_i = 1:length(r)
            h_el = r(r_i)*p(r_i).^n;
            h_partial = h_partial+h_el;
        end
        
        stem(n, h_partial);
    \end{lstlisting}

    \subsubsection{From z transfer function to time domain}
    Givan a LTI system:
    $$
    H(z)=\frac{
        z^{-5}+z^{-4}-3z^{-3}-8z^{-2}+7z^{-1}+9
    }{
        z^{-3}-2z^{-2}-z^{-1}+2
    }
    $$
    Find $h(n)$ in $n=0:100$
    \begin{lstlisting}[language=Matlab, escapeinside=`']
        A_z = [2 -1 -2 1];
        B_z = [9 7 -8 -3 1 1];
        n = 0:100;
        
        delta = zeros(1, length(n));
        delta(1) = 1;
        h = filter(B_z, A_z, delta);
    \end{lstlisting}

    \subsubsection{Zero-pole plot}
    Given $y(n)=x(n)-bx(n-1)+ay(n-1)$, find the expression of $H(z)$ and plot it in the z-plane
    \begin{lstlisting}[language=Matlab, escapeinside=`']
        B_z = [1 -b];
        A_z = [1 -a];
        zplane(B_z, A_z);
    \end{lstlisting}

\pagebreak\subsection{DFT}

    \subsubsection{DFT H(f) (and H(z)) amplitude and phase}
    We are given:
    $$
    y(n)=-2y(n-1)-y(n-2)+x(n)+2\rho\cos(\theta)x(n-1)+\rho^2x(n-2)
    $$
    With $\rho=0.9$, $\theta=\pi/8$ and sequence defined for $n$ $[0,N-1]$, $N=1000$
    \begin{lstlisting}[language=Matlab, escapeinside=`']
        A_z = [1 2 1]; % denominator
        B_z = [1 2 * rho * cos(theta) rho^2];
        N = 1000;
        [H_omega_2pi, omega_2pi] = freqz(B_z, A_z, N, 'whole');

        % plot from -pi to pi, magnitude/amplitude NOT normalized
        figure;
        plot(omega_2pi - pi, fftshift(abs(H_omega_2pi)));
        xlabel('\omega'); title('Amplitude'); grid

        % amplitude in normalized frequencies in [0,1)
        figure;
        plot(omega/(2*pi), abs(Hap));

        % amplitude in normalized frequencies in [-0.5,0.5)        
        plot(omega/(2*pi) - 0.5, fftshift(abs(H)));

        % plot from -pi to pi, phase
        figure;
        plot(omega_2pi - pi, fftshift(angle(H_omega_2pi)));
        xlabel('\omega'); title('Phase'); grid
    \end{lstlisting}

    \subsubsection{DFT by hand with matrix}
    Given an $h$, compute the DFT as $H=W_N\cdot h$, sequence defined for $n$ $[0,N-1]$, $N=1000$
    \begin{lstlisting}[language=Matlab, escapeinside=`']
        N = 1e3;
        n = 0:N - 1;

        A_z = [1 2 1]; % denominator
        B_z = [1 2 * rho * cos(theta) rho^2];
        %% h(n)
        delta = zeros(size(n));
        delta(1) = 1;
        h = filter(B_z, A_z, delta);
        
        W = exp(-1i * 2 * pi / N * (n' * n));
        H_k = W * h';
    \end{lstlisting}

    \subsubsection{DFT/FFT with function}
    \begin{lstlisting}[language=Matlab, escapeinside=`']
        H = fft(h); % careful, if filter has feedback (is IIR) this is not really DFT, we are computing a truncated version
        h = ifft(H); 
    \end{lstlisting}

    \subsubsection{DFT plot: normalized frequency}
    Given FIR filter $h(n)=1$, $n$ in $[0,19]$, visualize the DFT in normalized frequency $[-0.5,0.5)$
    \begin{lstlisting}[language=Matlab, escapeinside=`']
        N = 19;
        n_h = 0:N - 1;
        h = ones(size(n_h));
        
        H_k = fft(h);
        
        freq_axis = 0:N - 1; % k in [0,1/N...,(N-1)/N]
        freq_axis = freq_axis / N; % or we can define it as 0:1/N:1 - 1/N;
        freq_axis = freq_axis - 0.5; % just subtract to shift it, as f normalliy in [0, 1)
        figure;
        stem(freq_axis, fftshift(abs(H_k)));
        grid
    \end{lstlisting}

    \subsubsection{DFT plot: frequency (in Hz)}
    Given a sinusoidal with frequency $2Hz$ and duration $1.3s$, sampled with $F_s=50$
    \begin{lstlisting}[language=Matlab, escapeinside=`']
        f0 = 2;
        duration = 1.3;
        Fs = 50;
        time = 0:1 / Fs:duration;

        s = cos(2 * pi * f0 * time);
        N = 50;

        % FFT of the first 50 samples
        S_f = fft(s(1:N));
        freq_axis = 0:Fs / N:Fs - Fs / N;
        figure;
        stem(freq_axis, abs(S_f));

        % DFT of entire signal
        S_complete = fft(s);
        N_complete = length(s);
        freq_axis_complete = 0:Fs / N_complete:Fs - Fs / N_complete;
        % if shift needed, do something like freq_axis = freq_axis-Fs/2;
        figure;
        stem(freq_axis_complete, abs(S_complete));
    \end{lstlisting}

    \subsubsection{Zero padding}
    Same signal of before
    \begin{lstlisting}[language=Matlab, escapeinside=`']
        % Pad the array h(n) with zeros until reaching 100 samples
        % Evaluate the DFT of the padded h(n) and visualize it
        % Are the two DFTs equal? Comment on the results
        
        num_zeros = 100 - N;
        h_pad = padarray(h, [0, num_zeros], 'post');
        H_pad = fft((h_pad));
        
        N_pad = length(h_pad);
        freq_axis_pad = 0:N_pad - 1;
        freq_axis_pad = freq_axis_pad / N_pad;
        freq_axis_pad = freq_axis_pad - 0.5;
        
        figure;
        stem(freq_axis_pad, fftshift(abs(H_pad)));
        grid
        
        % Why are the two DFTs different?
        % In the second case we have more samples (we added more samples)
        % When we add zeros it means we are windowing a signal with a rectangle
        % Windowing = multiplying in the time
        % In the Fourier domain we are convolving with a sort of sinc
        % As we have a delta convolved with a signal, we get the signal
        % Also as we introduce sinc behavior, due to summation of tails
        % of the sinc we don't have peaks exactly at expected points
    \end{lstlisting}

    \subsubsection{DFT-periodic signals, sinusoidal}
    Given a sinusoidal with frequency $2Hz$ defined over $N=50$ samples, sampled with $F_s=50$ and a second sinusoid with frequency $2.2Hz$, equal duration and sampling rate, compute the DFT.
    \begin{lstlisting}[language=Matlab, escapeinside=`']
        Fs = 50;
        N = 50;
        time = 0:1/Fs:1/Fs*(N - 1);
        f0 = 2;
        f1 = 2.2;
        s0 = cos(2 * pi * f0 * time);
        s1 = cos(2 * pi * f1 * time);        
        s = s0 + s1;
                
        % FFT of s
        S_f = fft(s);
        freq_axis = 0:Fs / N:Fs - Fs / N;
        
        figure; stem(freq_axis, abs(S_f));
        
        % It will show two peaks, not what we expected, we wanted 4 peaks
        % this is due to discontinuities
        % to see the correct peaks use zero padding
        % To pad N_pad=k*Period
        % P_samples_1 = Fs/f1 = 25
        % P_samples_2 = Fs/f0 = 50/2.2 which is NOT AN INTEGER
        % but no worries, we can find the global period as well with lcm of
        
        N_tot = lcm(25, 250);
        num_pad = N_tot - N;
        s_pad = padarray(s, [0, num_pad], 'post');
        
        S_pad = fft(s_pad);
        freq_axis_pad = 0:Fs / N_tot:Fs - Fs / N_tot;
        
        figure; stem(freq_axis_pad, abs(S_pad));

        % But the plot seems a single sinusoid, zero padding does not help here!
        % Solution: increase #samples, that must be a multiple of the two periods
        % P_samples_1 = Fs/f1 = 25
        % P_samples_2 = Fs/f0 = 50/2.2
        
        N = 250; % == lcm(25,250)
        time = 0:1/Fs:1/Fs*(N - 1);
        s0 = cos(2 * pi * f0 * time);
        s1 = cos(2 * pi * f1 * time);        
        s = s0 + s1;
        S_f = fft(s);
        freq_axis = 0:Fs/N:Fs-Fs/N;
        
        figure; stem(freq_axis, abs(S_f));
    \end{lstlisting}

    \subsubsection{Cyclic convolution}
    \begin{lstlisting}[language=Matlab, escapeinside=`']
        x = [0 0 1 0 0]; n_x = 0:4;
        y = [5 4 3 2 1]; n_y = 0:4;
        % amount of samples over which we want to compute the cyclic conv
        N = length(x);

        % linear conv
        z = conv(x, y);
        n_z = n_x(1) + n_y(1):n_x(end) + n_y(end);

        % cyclic conv
        z_c = cconv(x, y, N);
        stem(0:N - 1, z_mat, '--');
    \end{lstlisting}   

\pagebreak\subsection{Filters}
    \subsubsection{A and B from z-transform filter: conv}
    Given the filter:
    $$
    H(z)=\frac{
        2(1-z^{-1})(1+0.5z^{-1})
    }{
        (1-0.8e^{j\pi/4}z^{-1})(1-0.8e^{-j\pi/4}z^{-1})
    }
    $$
    Derive $A(z)$ and $B(z)$
    \begin{lstlisting}[language=Matlab, escapeinside=`']
        B = conv(2*[1, -1], [1, 0.5]);
        A = conv([1, -0.8*exp(1i*pi/4)], [1, -0.8*exp(-1i*pi/4)]);
    \end{lstlisting}

    \subsubsection{Allpass filter}
    Define an allpass function
    \begin{lstlisting}[language=Matlab, escapeinside=`']
    function [z_out, p_out, b_out, a_out] = allpass(b, a)
        % Input: b, a = numerator and denominator of H(z)
        % Output: z_out, p_out, b_out, a_out = zeros, poles, numerator,
        % denominator of the allpass transfer function related to H(z)

        %%  first possibility: create Hap as H(z) / tilde{H(z)}

        % do this for both numerator and denominator
        a_tilde = fliplr(conj(a));
        b_tilde = fliplr(conj(b));

        b_out = conv(b, a_tilde);
        a_out = conv(a, b_tilde);

        % usually, every filter is normalized such that a_0 = 1
        % (but it's just a scaling operation)
        b_out = b_out / a_out(1);
        a_out = a_out / a_out(1);

        z_out = roots(b_out);
        p_out = roots(a_out);
    end
    \end{lstlisting}

    \subsubsection{Stability check}
    For a filter to be stable, all poles must be inside the unit circle
    \begin{lstlisting}[language=Matlab, escapeinside=`']
        stability_check = all(abs(p_ap) < 1);
        % or (~ is logical complement)
        stability_check = ~any(abs(p_ap) > 1);
    \end{lstlisting}

    \subsubsection{Minimum and maximum phase check}
    For a filter to be maximum phase, at least one zero is outside the unit circle
    \begin{lstlisting}[language=Matlab, escapeinside=`']
        function [filter_type] = typeOfFilter(b, a)
            % compute zeroes and poles
            zeroes = roots(b);
            poles = roots(a);
        
            % stability is related to poles only
        
            if any(abs(poles) > 1)
                filter_type = -1;
            else
                % the system is stable
                % check if it is minimum phase
                % also zeros must be inside the unit circle
                if any(abs(zeroes) > 1)
                    % the system is not minimum phase
                    filter_type = 0;
                else
                    % the system is minimum phase
                    filter_type = 1;
                end        
            end
        end
    \end{lstlisting}

    \subsubsection{Minimum phase-allpass decomposition}
    Properties of allpass: \textbf{Any rational causal stable system can be decomposed into the multiplication of a minimum phase system and an allpass system}
    $$
    H(z)=H_{min}(z)H_{ap}(z)
    $$
    Causal and stable, so all poles are inside the unit circle!\\
    $H_{min}(z)$ contains:
    \begin{itemize}
        \item The poles and the zeros of $H(z)$ that lie inside the unit circle
        \item Zeros that are conjugate reciprocals of the zeros $H(z)$ lying outside the unit circle
    \end{itemize}
    $H_{ap}(z)$ contains:
    \begin{itemize}
        \item All the zeros of $H(z)$ that lie outside the unit circle
        \item Poles that are conjugate reciprocals of the zeros of $H(z)$ lying outside the unit circle
    \end{itemize}
    \begin{lstlisting}[language=Matlab, escapeinside=`']
        B = [1, -1.98, 1.77, -0.17, 0.21, 0.34];
        A = [1, 0.08, 0.40, 0.27];
        
        % all-pass minimum phase decomposition
        zeroes = roots(B);
        poles = roots(A);
        
        % zeros and poles which are already minimum phase
        z_min = zeroes(abs(zeroes) <= 1);
        p_min = poles(abs(poles) <= 1);
        
        % check if there are zeroes outside the unit circle, maximum phase
        if any(abs(zeroes) > 1)
            % allpass decomposition
            z_ap = zeroes(abs(zeroes) > 1); % extract the maximum phase zeroes
            p_ap = 1 ./ conj(z_ap); % poles of allpass compensate for the zeros
        
            % minimum phase part
            % include other zeros = p_ap in the minimum phase filter
            z_min = [z_min; p_ap];
        
            % define the polynomials related to the all pass filter
            B_ap = poly(z_ap);
            A_ap = poly(p_ap);
        else %system is minimum phase
            B_ap = 1;
            A_ap = 1;
        end
        
        % define the polynomials related to the minimum phase filter
        B_min = poly(z_min);
        A_min = poly(p_min);
        
        % multipy by c_0 if want amplitude of allpass = 1
        c_0 = sum(A_ap) / sum(B_ap);
        B_ap = c_0 * B_ap;
        
        % divide Hmin by c_0, so to compensate the multiplication before
        B_min = B_min / c_0;
        
        %% check on zplane
        figure; zplane(B_min, A_min); grid; title("Minimum phase component");        
        figure; zplane(B_ap, A_ap); grid; title("Allpass component");
    \end{lstlisting}

    \subsubsection{FIR filter to attenuate signal and conjugate zero/pole}
    You are given two zeros with absolute value equal to 2 in complex conjugate position, build a FIR filter in order to attenuate a signal with frequency $f_1=\cdots$
    \begin{lstlisting}[language=Matlab, escapeinside=`']
        z1 = 2*exp(1i*2*pi*f1);
        z2 = conj(z1);

        A = 1; % it is a FIR
        % If we have two zeroes which are complex conjugate, we can
        % always define the polynomial related to the zeros as
        % 1 - 2*rho*cos(theta)z^{-1} + rho^2 z^{-2}
        B = [1, -2*cos(2*pi*f1), 4];
        y = filter(B,A,x);
    \end{lstlisting}

    \subsubsection{Windowing}
    Given $x$ as a consine wave sampled at $F_s=8kHz$, defined from 0 to 1 second, amplitude $1.5$, frequency $1.1kHz$, phase $45 deg$. Compute $y$ as $x$ filtered with a low-pass filter with normalized cutoff frequency of $0.4$ and 64 samples.
    \begin{lstlisting}[language=Matlab, escapeinside=`']
        Fs = 8e3; duration = 1; A = 1.5; f0 = 1.1e3; phase = pi/4;
        time = 0:1/Fs:duration;

        N_filter = 64;
        filter_order = N_filter - 1;
        cutoff = 0.4;
        cutoff_filter = cutoff * 2; % matlab cutoff always twice of desired

        h = fir1(filter_order, cutoff_filter);
        y = filter(h, 1, x); % as FIR, we put 1 as denominator
    \end{lstlisting}
    Apply a Hanning window to select the first 512 samples of $y$, then plot the DFT of the windowed $y$ vs frequency in Hz, defined in $[-F_s/2,F_s/2]$
    \begin{lstlisting}[language=Matlab, escapeinside=`']
        Nw = 512;
        w = window(@hann, Nw); % or hann(Nw);
        y_w = y(1:Nw) .* w'; % multiply element wise
        
        %% we can try to exam some common windows behaviour
        w1 = rectwin(64);
        w2 = bartlett(64); % triangular
        w3 = hamming(64);
        w4 = blackman(64);
        % open window tool
        wvtool(w1, w2, w3, w4);
        
        % compute DFT
        Yw = fft(y_w);
        freq_axis = 0:Fs/Nw:Fs-Fs/Nw;
        freq_axis = freq_axis-Fs/2;
        figure;
        plot(freq_axis, Yw);
    \end{lstlisting}
    
\pagebreak\subsection{Multirate}
    \subsubsection{Downsampling and decimation}
    Given a signal $x(n)$, downsample with factor $M=4$, decimate it with a decimation factor of $M=4$, using a FIR filter with order 64.
    \begin{lstlisting}[language=Matlab, escapeinside=`']
        Fs = 500; f_0 = 50; f_1 = 100; duration = 3; time = 0:1 / Fs:duration;
        x = cos(2 * pi * f_0 * time) + cos(2 * pi * f_1 * time);

        M = 4;

        % LPF
        cut_off = 1 / (2 * M);
        cut_off_filter = 2 * cut_off;
        lpf = fir1(64, cut_off_filter);

        x_f = filter(lpf, 1, x); % FIR

        % decimate
        x_decimated = x_f(1:M:end); % just specify a step size of M

        % As original pulse in 50 Hz, after downsampling with M = 4
        % we will find a pulse/peak in 4*50=200. Same goes for 100*4 = 400
        % Without decimation we have aliasing at freq 100 (and its symmetric 400)
        % With lpf we will not have contributions of 100 and 400, as cutoff was
        % 1/(2M) = 1/8. BUT as we are in Hz, 1/8*Fs=1/8*500 = 62.5, so any freq
        % higher than that removed (100 > 62.5 removed, so its symmetric removed as well)
        % BM < Fs/2 -> 50*4 < 250 yes, but 100*4 < 250 NO
    \end{lstlisting}

    \subsubsection{Upsampling and interpolation}
    Given the downsampled signal from before, without lpf first, decimate it with a decimation factor of $M=4$, using a FIR filter with order 64.
    \begin{lstlisting}[language=Matlab, escapeinside=`']
        % Starting signal, from before
        Fs = 500; f_0 = 50; f_1 = 100; duration = 3; time = 0:1 / Fs:duration;
        x = cos(2 * pi * f_0 * time) + cos(2 * pi * f_1 * time);
        M = 4;
        x_downsampled = x_f(1:M:end); % starting signal, applied downsample without lpf

        % Upsample it WITHOUT applyting lpf at the end
        L = 4;
        x1 = zeros(1, length(x_downsampled) * L);
        x1(1:L:end) = x_downsampled;
    \end{lstlisting}
    Given the same downsampled signal, but this time with lpf applied, upsample with interpolation
    \begin{lstlisting}[language=Matlab, escapeinside=`']
        % Starting signal, downsampled signal with lpf this time
        cut_off = 1 / (2 * M);
        cut_off_filter = 2 * cut_off;
        lpf = fir1(64, cut_off_filter);
        x_f = filter(lpf, 1, x);
        x_decimated = x_f(1:M:end);

        % Upsample it and then apply lpf
        L = 4;
        x2 = zeros(1, length(x_decimated) * L);
        x2(1:L:end) = x_decimated;

        lpf = fir1(64, 1/L); % cut_off = 2*(1/(2*L))=1/L
        % remember to put the gain L in interpolation
        x2 = L*filter(lpf, 1, x2); % MULTIPLY AGAIN BY L

        % at the end, I WILL NOT OBTAIN THE ORIGINAL SIGNAL, i lost info
        % as I downsampled to avoid aliasing (aliasing makes me lose information)
        % At the end only in f0 we will find a sinusoidal contribution
        % we also will find a delay at the beginning (many zeros at the beginning)
        % This is because every filter in real life is a causal filter, so in the final
        % output will be delay (phase related with delay) -> lpf's introduce delay
        % x(n) -> [LPF*LPF] -> y(n)
        % The convolution of those two LPF's introduce delay
        % If we inspect the maximum peak in freq domain of the convolution/mutiplication
        % as the peak is not in 0, but e.g. 65, a delay will be introduced
        % To find the delay: 
        % [max_filter, filter_delay] = max(conv(lpf,lpf));
        % filter_delay is the position of the max, so delay
        % So to delete delay:
        % actual_signal = cos(2*pi*f*(time(1:N)-time(filter_delay)));

        [max_filter, filter_delay] = max(conv(lpf, lpf)); % lpf_up == lpf_down in this case
        actual_signal = cos(2*pi*f_0 * (time(1:N) - time(filter_delay)));
    \end{lstlisting}

    \subsubsection{Rational sampling rate conversion: noninteger factor}
    Given the signal $x(t)=A_1\cos(2\pi f_1t)+A_2\cos(2\pi f_2t)$, create the signal $x(n)$ as $x(t)$ from 0 to 0.5 seconds, sampled at $F_s=8000Hz$, $A_1=0.7$, $A_20+=0.5$, $f_1=1800Hz$, $f_2=3600Hz$. Then create the signal $y(n)$ by resampling $x(n)$ with $6000Hz$, suing $N=64$ filter samples;
    \begin{lstlisting}[language=Matlab, escapeinside=`']
        duration = 0.5; Fs = 8e3; time = 0:1/Fs:duration;
        A1 = 0.7; f1 = 1800;
        A2 = 0.5; f2 = 3600;

        x = A1*cos(2*pi*f1*time) + A2*cos(2*pi*f2*time);

        % 8000 -> 24000 -> 6000
        L = 3; M = 4; % or use [L, M] = rat(Fs_new / Fs); 

        % Upsample
        x_upsampled = zeros(1, length(x) * L);
        x_upsampled(1:L:end) = x;

        % LPF with cutoff of min(1/(2L),1/(2M))
        cutoff = min([1/(2*L), 1/(2*M)]);
        cutoff_filter = 2*cutoff;
        h = fir1(64-1, cutoff_filter);
        x_f = L*filter(h, 1, x_upsampled);

        % Downsample
        y = x_f(1:M:end);
    \end{lstlisting}

\pagebreak\subsection{Functions Recap}
\begin{lstlisting}[language=Matlab, escapeinside=`']
    %% Shifting discrete signals, a positive value will shift to the right. Circular, if shifting right by n, first n values will become the last n values
    circshift([1 2 3 4 5], 2);
    % [4 5 3 2 1]


    %% Periodic sequence generation. The first paramater is the matrix, the second one is the rows repetition, the third is the cols repetition
    repmat([1 2 3],1,3);
    % [1 2 3 1 2 3 1 2 3]
    repmat([1 2 3],2,3);
    % [1 2 3 1 2 3 1 2 3;
    %  1 2 3 1 2 3 1 2 3] 


    %% Returns the convolution of vectors u and v. If u and v are vectors of polynomial coefficients, convolving them is equivalent to multiplying the two polynomials.
    conv([1 1],[1 1]); % (1+x)(1+x)
    % [1 2 1], which is (1+x)(1+x)=1+2x+x^2


    %% Roots of a polynomial
    roots([1 0 1]); %1+x^2
    % [0.0000+1.0000i, 0.0000-1.0000i]'


    %% Partial fract expansion
    A_z = [2 -1 -2 1];
    B_z = [9 7 -8 -3 1 1];
    [r , p , c ] = residuez(B_z, A_z);


    %% Apply the filter of numerator B_z, denominator A_z to input delta
    h = filter(B_z, A_z, delta);


    %% z-plane, zero-pole plot
    zplane([1 -0.5],[1 -0.2]);
    % A zero in 0.5 and a pole in 0.2


    %% DFT magnitude and phase plot
    [H_omega_2pi, omega_2pi] = freqz(B_z, A_z, N, 'whole');
    % plot from -pi to pi
    plot(omega_2pi - pi, fftshift(abs(H_omega_2pi)));
    plot(omega_2pi - pi, fftshift(angle(H_omega_2pi)));


    %% DFT/FFT and inverse
    H = fft(h);
    h = ifft(H)


    %% Pad an array
    padarray([1 1], [0, 3], 'post');
    % [1 1 0 0 0]
    padarray([1 1], [2, 1], 'post');
    % [1 1 0; 0 0 0; 0 0 0]


    %% Cyclic convolution
    cconv(x, y, N);


    %% Invert an array
    fliplr([1 2 3 4])
    % [4 3 2 1]

    
    %% Get numerator and denominator of a fraction
    [L, M] = rat(6000 / 8000); 
    % [3, 4]
\end{lstlisting}
