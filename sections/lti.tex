% !TeX root = ../main.tex

\section{Linear Time-Invariant Systems}
\begin{itemize}
    \item Linearity = input-output realtionship is linear
    \item Time invariance = the output does not depend on the particular time the input is applied:
    $$x(t)\rightarrow y(t)\leftrightarrow x(t-k)\rightarrow y(t-k)$$
    \item The system can be completely characterized by its impulse response $h(t)$
\end{itemize}

The output:
\begin{LARGE}
    $$
    y(n)=x(n)*h(n)=\sum_{k=-\infty}^\infty x(k)h(n-k)
    $$
\end{LARGE}

\subsection{Convolution}
\begin{itemize}
    \item Flip the second term $h(n)\rightarrow h(-k)$
    \item Shift $h$ by adding $n$ (if positive shift to right, if negative to left) $h(n-k)$
    \item For output $y(n)$ sum all contributions of $x(k)$ and the shifted flipped $h$
\end{itemize}
Last step similar to scalar product. \textbf{The length of the convolution is the sum of the two signals-1 (n+m-1) and the support (x-axis from when the signals start to differ from 0) is the sum of the supports until the final support of the two}.

Properties of convolution:
\begin{itemize}
    \item Commutativity
    \item Associativity
    \item Distributivity
    \item Convolution by pulse $x(n)*\delta(n)=x(n)$
    \item Convolution by a shifted pulse $x(n)*\delta(n-k)=x(n-k)$
\end{itemize}

Alternativetely, if $h(n)=h[n]=$sum of various $\delta$, we can apply the distributive property and sum various convolutions by $\delta$!

\subsection{Discrete delay}
$$y(n)=x(n-k)$$
So:
$$y(n)=x(n)*h(n)=x(n)*\delta(n-k)=x(n-k)$$

\subsection{Moving average}
\begin{LARGE}
    $$
    y(n)=\frac{1}{M}\sum_{m=0}^{M-1}x(n-m)
    $$
\end{LARGE}
It is a filter, result of LTI system.

The \textbf{impulse response}:
$$y(n)=\frac{1}{M}\sum_{m=0}^{M-1}x(n)*\delta(n-m)=x(n)*\sum_{m=0}^{M-1}\delta(n-m)\rightarrow \mathbf{\sum_{m=0}^{M-1}\delta(n-m)}$$

\subsection{Cosine signal and generic filter}
We will get another cosine with very same frequency, but modified amplitude and phase. \textbf{Look at the exercise 20181107-1}